\documentclass[a4paper,titlepage,openany,12pt]{report}

\usepackage[T1]{fontenc}
\usepackage[utf8]{inputenc}
\usepackage[francais]{babel}
\usepackage{lmodern}
\usepackage{pxfonts}
\usepackage{graphicx}

\title{Projet CPS (MI047) - River City Ransom}
\author{Kevin Coquart \and Quentin Bunel}

\begin{document}

\maketitle

\section*{Introduction}

L'objectif du projet est de réaliser la spécification du jeu River City Ransom,
puis d'implémenter celle-ci (décorateurs, contracts, tests) selon la méthodologie du cours.
Nous avons découpé le travail en 11 services.

Les spécifications ont été redigées dans le format org. C'est un format de l'éditeur Emacs
qui permet de générer facilement des fichiers html ou pdf (voir plus bas).

\section*{Solutions}

\paragraph{}
Un des premiers problèmes qui s'est présenté à nous est qu'il fallait différencier les objets équipables
des objets marchands. En effet, les objets marchands ont un prix et peuvent se vendre, alors que les objets 
équipables doivent contenir un bonus, c'est-à-dire un nombre de points de vie à retirer au personnage cible 
d'un éventuel jet. Notre service Objet déclare uniquement un observateur nom() de type String et possède deux 
sous-services : ObjetEquipable et ObjetMarchand.

Un autre facteur vient cependant modifier la solution : un personnage peut être porté par un autre !
Personnage raffinerait donc ObjetEquipable ? et donc Objet ?
La solution que nous avons choisi a été la création d'un service Chose qui décrit ce qui est portable et 
le bonus apporté. Personnage et ObjetEquipable raffinent tous les deux Chose.
Ainsi, quand on a besoin de manipuler des objets, comme dans les blocs du terrain, on est sûr de ne pas
confondre avec les personnages.

\\ inclure le scheme de classe

\paragraph{}
Pour avoir accès aux personnages, dans GestionCombat, nous avons fait le choix d'avoir une Map dont les
clés sont les noms des personnages, et les valeurs les instances de Personnage (ou Gangster). On peut par 
exemple appeler mPerso.get("Alex") pour récupérer Alex.
De plus, la méthode gerer() prend en paramètre une Map<String, COMMANDE> qui associe à chaque personnage
la commande qu'il doit réaliser dans le pas de jeu.
\\ rajout verif ortho
Cela nous a permit d'unifier le traitement des differents types de
pions (Alex, Ryan, Slick et les gangsters).
Grace a leur nom, on a acces a l'objet qui les representes, leur
position, et leur etat (gel).
Au total, nous gerons 3 Map pour l'objet, la position et leur etat.

\\ rajout
\paragraph{}
Nous avons decidé de faire de gestion combat un processus du jeu mais
de ne pas lui donner de pouvoir decission, les decissions viennent du
moteur de Jeu.
Le moteur de jeu donne les commandes pour tout le monde, bot compris.
Il peut aussi ajouter un nouveau nom a la liste des commandes et ainsi
creer un nouveau gangster.
Ce mode de gestion separe permet de bien differencier le centre de
decision du processus d'avancement. Cela simplifie aussi les
ameliorations possible.
En modifiant moteur jeu l'on peut aisement faire une nouvelle
implémentation ou les gagnsters deviendrait controlable. Et pourquoi
pas en faire une version reseau avec 2 camps, les gentils et les mechants.


\section*{Annexes}

Vous trouverez ci-joint :
\begin{itemize}
\item Dans /spec, les specifications au format org et le Makefile pour générer les formats html et pdf (voir README.txt)
\item Dans /tests, les tests au format org et le Makefile pour générer les formats html et pdf (voir README.txt)
\item Le code des services, décorateurs, contracts, tests et implémentation dans /src
\end{itemize}

\section*{Version Bogue}
\paragraph{}
Il y a deux implementation bogue, celle de personnage dont les
contratcs renvoient de joli erreur.
Et celle de gestion combat, pour celle si, le contrat ne pouvant
verifier les passage en erreur ne produit pas d'exception par contre
les tests renvoie de joli message indiquant qu'il y a un soucis.



\section*{Conclusion}

\includegraphics[scale=0.3]{../saved}

\\
\paragraph{}
Le jeu est fonctionnel, il n'est surement pas exempt de bogue. Nous
avons essaye plusieurs scenario sans aucun probleme. L'interface
graphique est assez minimaliste, quelques ameliorations possibles :
\begin{itemize}
\item l'ajout du pas de jeu dans une barre d'etat
\item une boite d'alerte avec le resultat final
\item la gestion de l'affichage des points de vie
\end{itemize}
Bien sur les graphismes ne sont pas fou, des petits carre de couleur
diverse et varie mais l'ensemble reste coherant et comprehensible.
Le jeu est entierement jouable au clavier.


\\
\section*{Notice}
\paragraph{}
Le jeu se partage en 3 mode de jeu
\begin{itemize}
\item automatique
\item un joueur
\item deux joueurs
\end{itemize}

\paragraph{}
Pour changer le comportement des personnages non controle il faut
aller voir la classe tools.GenerateurCmd .
Il existe deux codes pour la meme methode, une version intelligent qui
va plus souvent a se deplacer puis frapper de temps a autre, ainsi
qu'une version beaucoup plus simple. Dans cette derniere, on peut
basculer en commentant/decommantant dans un mode ou tout le monde est
beau, tout le monde est gentil, la commande renvoye est RIEN. Ou bien
c'est un random sur un tableau de commande donne, a nous de choisir
les commandes du tableau.

\paragraph{}
Pour jouer, il suffit de choisir son mode de jeu, le mode 1 jouer est
selectionne au depart. Puis choisir la commande a effectuer (clavier
ou souris).
En mode deux joueurs, il faut choisir deux commandes, l'une a la suite
de l'autre.


\section*{Connaissance acquise}
\paragraph{}
Ce projet nous a permit de decouvrir un aspect qui nous etait meconnu
jusqu'a present, le monde des specifications.
Au fil de notre scolarite, nous decouvrons l'aspect projet petit a
petit.
Au debut sur des projets simple, nous implementions la solution sans
analyse prealable puis par la suite avec l'ecriture des algorithme
complique au prealable pour savoir sur quel base partir.
Ensuite, nous avons ajoute l'UML a nos competances, un travail plutot
long et fastudieux pour le gain qu'il apporte. Les outils a notre
disposition a la fac ne nous permettant pas de gagner reelement de
temps en passant par ces structures.
Enfin grace a cet UE nous avons decouvert les specifications.

\paragraph{}
Creer des specifications pour chaque services puis des tests sur
chacun est aussi un travail assez fastudieux mais bien plus complexe
est complet. Par la suite la plupart de l'implementation est trivial.
Les tests permettent par la suite de verifier son travail et contaster
les erreurs pour les corrigers.
Une fois ces erreurs resolu, la mise en commum des sevices et l'essai
du jeu se sont passe sans soucis.

\paragraph{}
Nous retenons plusieurs chose de ce projet, premierement, il est
difficile de faire un jeu en 3 semaines.
Deuxiement, nous ne sommes pas graphiste et celle se remarque tres
vite.
Pour finir nous avons decouvert le monde des specifications avec une
approche proche de l'industrie, l'on produit une specification
complete. Puis quelqu'un se charge de l'implementation et pour finir
avec nos test on valide son travail.
Bien sur dans le cadre de ce projet, l'implementation est faite par
nous meme, cela d'ailleurs simplifie grandement la comprehension des
specifications.


\end{document}
